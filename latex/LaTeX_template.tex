\documentclass[11pt]{article}
\usepackage{amsmath,amssymb,amsthm}
\addtolength{\evensidemargin}{-.5in}
\addtolength{\oddsidemargin}{-.5in}
\addtolength{\textwidth}{0.8in}
\addtolength{\textheight}{0.8in}
\addtolength{\topmargin}{-.4in}
\newtheoremstyle{quest}{\topsep}{\topsep}{}{}{\bfseries}{}{ }{\thmname{#1}\thmnote{ #3}.}
\theoremstyle{quest}
\newtheorem*{definition}{Definition}
\newtheorem*{theorem}{Theorem}
\newtheorem*{question}{Question}
\newtheorem*{exercise}{Exercise}
\newtheorem*{challengeproblem}{Challenge Problem}
\newcommand{\name}{%%%%%%%%%%%%%%%%%%
%%%%%%%%%%%%%%%%%%%%%%%%%%%%%%
%%%%%%%%%%%%%%%%%%%%%%%%%%%%%%
%% put your name here, so we know who to give credit to %%
[Your Name Here]
}%%%%%%%%%%%%%%%%%%%%%%%%%%%%%%
\newcommand{\hw}{%%%%%%%%%%%%%%%%%%%%
%% and which homework assignment is it? %%%%%%%%%
%% put the correct number below              %%%%%%%%%
%%%%%%%%%%%%%%%%%%%%%%%%%%%%%%
1
}
%%%%%%%%%%%%%%%%%%%%%%%%%%%%%%
%%%%%%%%%%%%%%%%%%%%%%%%%%%%%%
%%%%%%%%%%%%%%%%%%%%%%%%%%%%%%
\title{\vspace{-50pt}
\Huge \name
\\\vspace{20pt}
\huge POLSCI 699\hfill Problem Set \hw}
\author{}
\date{}
\pagestyle{myheadings}
\markright{\name\hfill Problem Set \hw\qquad\hfill}

%% If you want to define a new command, you can do it like this:
%%This function is quite useful
\newcommand{\Q}{\mathbb{Q}}
\newcommand{\R}{\mathbb{R}}
\newcommand{\Z}{\mathbb{Z}}
\newcommand{\C}{\mathbb{C}}

%% If you want to use a function like ''sin'' or ''cos'', you can do it like this
%% (we probably won't have much use for this)
% \DeclareMathOperator{\sin}{sin}   %% just an example (it's already defined)


\begin{document}
\maketitle

You can start writing in this file, but first save it as ``pset1.tex'' or something, and put your name into the file(up near the top, where all the percent signs are). 

\begin{question}[1]
  It's a good idea to restate the question here.
\end{question}
\begin{proof}
  Insert your proof here.
\end{proof}

\begin{question}[2]

\end{question}
\begin{proof}

\end{proof}

\begin{question}[2b]

\end{question}
\begin{proof}

\end{proof}

There are similar environments for Theorems.

\begin{theorem}[3]
  You should restate the theorem here.
\end{theorem}
\begin{proof}
  Insert your proof here.
\end{proof}


\section*{Writing in math mode}
When you want to write something in ``math mode'', you should enclose it in dollar signs: $x+y=a^2+b^2$.
If it's an important equation and you want to set it off, you can do it like this: \[x = \frac{-b \pm \sqrt{b^2-4ac} }{2a}\]
If you have a chain of deductions, you can line them up like this. (The ampersand \& tells it where to line up---usually you will want it right before the equals sign. You should have an ampersand in each line. The double-backslash $\backslash\backslash$ tells it to move to the next line.)
\begin{align*}
  2x &= 2y-10&\\
  x &=y-5 &\\
  z+x &= y-5+z
\end{align*}
If you want to include justifications, one way is like this:
\begin{align*}
  2x &= 2y-10 & (A5)\\
  x &=y-5  & (A4)\\
  z+x &= y-5+z & (W)
\end{align*}
You can also include words:

\begin{align*}
  2x &= 2y-10 & \text{by commutativity} \\
  x &=y-5  & \text{since $y$ is the GCD}\\
  z+x &= y-5+z & \text{because 2 is prime}
\end{align*}

Here are some symbols that might come up in the homeworks and journals.

Let $x\in \Z[i]$ and $y\in \Q[i]$. Then $x^2+y^2\in \Q[i]$. If you have two numbers $n_1\in \Z$ and $n_2\in \Z$, then you should be satisfied. You can ask whether $n_1 < n_2$, or even if $n_1 \leq n_2$. I wonder if $a|b$? Greek letters: $\alpha$, $\beta$, $\gamma$, $\pi$, $\Gamma$, $\Delta$.

Maybe you want to talk about the set $S=\{ x\in \Z | x^2 = 2\}$, or maybe the set $T=\{ x\in \Z | x^2 \text{ is even}\}$. Did you notice that $S\subseteq T$? It's also true that $S \neq T$.

You might need to write fractions like $\frac{p}{q}$, or even $\frac{a}{a^2+b^2}$. If this fractions are too small, you can display them like this (see how I made those big parentheses?):
\[\left(\frac{p}{q}\right)\cdot \left(\frac{t}{u}\right) = \frac{a}{a^2+b^2}\]
If you can't figure out a symbol, you should google ``Detexify''. All you have to do is draw the symbol using your mouse, and it tells you the latex command.
%%%% don't delete the last line!
\end{document}
