%% Based on Overleaf Sample Paper template

\documentclass[a4paper, uplatex]{jsarticle} %Japanese allowed
%% Language and font encodings
\usepackage[english]{babel}
\usepackage[utf8x]{inputenc}
\usepackage[T1]{fontenc}

%% Sets page size and margins
\usepackage[a4paper,top=3cm,bottom=2cm,left=3cm,right=3cm,marginparwidth=1.75cm]{geometry}

%% Useful packages

% Math font
\usepackage{amsmath}
\usepackage{amssymb}
\usepackage{amsfonts}

% graph & pictures
\usepackage{graphicx}



\title{Latent Dirithlet Allocation}
\author{Saki}



%Document starts here

\begin{document}
\maketitle

\begin{abstract}
\end{abstract}

\section{Reference}
Blei, D.M., Ng, A.Y. and Jordan, M.I., 2003. Latent dirichlet allocation. Journal of machine Learning research, 3(Jan), pp.993-1022.

\section{Model}
Generative Process for each word 

\begin{align*}
\theta &\sim Dir(\alpha) \\
z_n &\sim Multinomial(\theta) \\
w_n &\sim p(w_n|z_n, \beta)
\end{align*}

Joint Probbailities of $\theta, z_n, w_n$ given $\alpha, \beta$
$$
p(\theta, Z, W| \alpha, \beta) = p(\theta | \alpha) \prod_{n=1}^N p(z_n|\theta) p(w_n|z_n, \beta)
$$

\section{Intuition}
Imagine a big box, which corresponds to a document. We put small boxes, which corresponds to topics, into the big box.\\ 
The size of the small boxes are determined by $\theta$ (e.g. Sports box takes 50\% of the big box, Politics box 20\%, and Economics box 30\%). ($p(\theta|\alpha)$) \\
Once the small box are fit, we chose one small box($p(z_n|\theta)$), and throw a ball (i.e. a word) into the chosen small box.($p(w_n|z_n,\beta)$) \\
This means that we pick a word which is likely to occur when we write about the chosen topic (e.g. 'baseball' in sport topic), and write down the word on the document. \\
Repeat this process for the number of words in the document. 

\end{document}

